%!TEX root = ../thesis.tex
%*******************************************************************************
%*********************************** First Chapter *****************************
%*******************************************************************************
\fancypagestyle{plain}{%              
\fancyhf{}
%\renewcommand{\subsectionmark}[1]{%
  %\markright{\MakeUppercase{\thesubsection\ \ \ \ #1 }}}
  %\renewcommand{\chaptermark}[1]{\markboth{teste ##1}{}}
  %\renewcommand{\sectionmark}[1]{\markright{\thesection\ ##1}}

  %\renewcommand{\headrulewidth}{0.7pt}
    %\fancyhead[LO]{\bfseries  \nouppercase{CAPÍTULO \thechapter}  }
      \renewcommand\footrule{%

 \color{\corcaps}\noindent\makebox[\linewidth]{\rule{\paperwidth}{1pt}}
}

      %\renewcommand\headrule{%

 %\color{\corcaps}\noindent\makebox[\linewidth]{\rule{\paperwidth}{1pt}}
%}
        %\fancyhead[LO]{\bfseries}
        \fancyfoot[RO]{\bfseries Página \thepage \ de \pageref*{LastPage}}
    %\fancyhead[RO]{\bfseries \nouppercase \leftmark}
    }
    
\fancypagestyle{MyFancy}{%              
\fancyhf{}
%\renewcommand{\subsectionmark}[1]{%
  %\markright{\MakeUppercase{\thesubsection\ \ \ \ #1 }}}
  %\renewcommand{\chaptermark}[1]{\markboth{##1}{}}
  %\renewcommand{\sectionmark}[1]{\markright{\thesection\ ##1}}
  %\renewcommand{\headrulewidth}{0.7pt}
  \renewcommand{\sectionmark}[1]{ \markright{ ##1}}
  \renewcommand\headrule{%

 \color{\corcaps}\noindent\makebox[\linewidth]{\rule{\paperwidth}{1pt}}
}
  \renewcommand\footrule{%

 \color{\corcaps}\noindent\makebox[\linewidth]{\rule{\paperwidth}{1pt}}
}
	%\fancyhead[RO]{\bfseries \rightmark}
	%\fancyhead[LO]{\bfseries \nouppercase{\thesubsection} }
    \fancyhead[RO]{\bfseries \nouppercase{\rightmark}  
    }
    \fancyfoot[RO]{\bfseries Página \thepage \ de \pageref*{LastPage}}
    \fancyfoot[LO]{\bfseries }
    %\fancyhead[RO]{\bfseries \nouppercase \leftmark}
    }
\addtocounter{page}{9}
\pagestyle{MyFancy}

\color{gal}
\chapter*{Introdução}\addcontentsline{toc}{chapter}{Introdução}\markboth{Introdução}{Introdução}
\label{cap:intReal}



\PlaceText{69mm}{35mm}{ \color{gal}\noindent\makebox[\linewidth]{\rule{2\paperwidth}{1pt}}}

\PlaceText{69mm}{63mm}{ \color{gal}\noindent\makebox[\linewidth]{\rule{2\paperwidth}{1pt}}}

\PlaceText{15mm}{12mm}{ \color{white}\noindent\makebox[\linewidth]{\rule{2\paperwidth}{10pt}}}

\iffalse
\color{red}

\begin{todolist}

\boldmath
    % pra marcar o item como feito é só colocar \item[\done]
    \bf{
    %\item[\done] ainda não feito
        \item[\done] terminar resumo
        \item[\done] palavras-chave
	\item[\done] terminar a introdução
	\item[\done] terminar de arrumar as referências
	\item arrumar pontuação das equações
	}
\end{todolist}




\fi

\color{black}
\lettrine[nindent=2em,lines=1]{E}m geral, ...
