%TODO: Make a revision in notation commands, they are very confusing...

% General


\DeclareFontFamily{U}{stix2bb}{}
\DeclareFontShape{U}{stix2bb}{m}{n}{<->stix2-mathbb}{}

\newcommand{\indic}{\text{\usefont{U}{stix2bb}{m}{n}1}}% indicator function


\newcommand{\p}{\partial}
\newcommand{\bee}{\widetilde{\be}}
\newcommand{\Rmm}[4]{\Rm\parent{#1,#2}#3, #4}
\newcommand{\segf}{\mathbf{\mathrm{\RNum{2}}}}
\newcommand{\RNum}[1]{\textbf{\uppercase\expandafter{\romannumeral #1\relax}}}
\newcommand{\conj}[2]{\{#1 \ \vert \ #2 \}}
\newcommand{\tioC}[2]{\widetilde{#1}^{#2}}
\newcommand{\tioB}[2]{\widetilde{#1}_{#2}}
\newcommand{\Munderbrace}[2]{\begingroup \color{violet} \underbrace{\color{black} #1 }_{\color{violet} #2 } \endgroup}
\newcommand{\Bunderbrace}[2]{\begingroup \color{gal} \underbrace{\color{black} #1 }_{\color{gal} #2 } \endgroup}
\newcommand{\Moverbrace}[2]{\begingroup \color{violet} \overbrace{\color{black} #1 }^{\color{violet} #2 } \endgroup}
\renewcommand{\l}{\ell}
\renewcommand{\iint}{\mathrm{int}}
\newcommand{\intprodl}{%
    \mathbin{\scalebox{1.5}{$\lrcorner$}}%
}
\theoremstyle{theoremTEO}
\newtheorem{thm}{Theorem}[section] 
\newcommand{\thistheoremname}{}
\newtheorem{genericthm}[thm]{\thistheoremname}
\newenvironment{namedthm}[1]
  {\renewcommand{\thistheoremname}{#1}%
   \begin{genericthm}}
  {\end{genericthm}}
  

\theoremstyle{theoremLEM}
\newtheorem{generic}[thm]{\thistheoremname}
\newenvironment{namedlem}[1]
  {\renewcommand{\thistheoremname}{#1}%
   \begin{generic}}
  {\end{generic}}
  

\newcommand{\bigzero}{\mbox{\normalfont\Large\bfseries 0}}
\newcommand{\rvline}{\hspace*{-\arraycolsep}\vline\hspace*{-\arraycolsep}}
\newcommand{\V}{\mathbb{V}}
\newcommand{\divv}{\operatorname{div}}
\newcommand{\RICC}{\operatorname{RICC}}
\newcommand{\dv}{\operatorname{dVol}_g}
\newcommand{\ddv}{\widetilde{\operatorname{dVol}_g}}
\newcommand{\SC}{\operatorname{SC}}
\newcommand{\End}{\operatorname{End}}
\newcommand{\dx}{\mathrm{d}x}
\newcommand{\dr}{\mathrm{d}r}
\newcommand{\dd}{\mathrm{d}}
\newcommand{\R}{\mathscr{R}}
\newcommand{\n}{\nabla}
\newcommand{\W}{\mathscr{W}}
\newcommand{\Sy}{\mathcal{S}}
\newcommand{\T}{\mathscr{T}}
\newcommand{\Endo}{\operatorname{End}}
\newcommand{\w}{\omega}
\newcommand{\signature}{\Sigma}             %The signature set.
\newcommand{\Id}{\mathrm{Id}}
\newcommand{\ee}{\mathcal{E}}
\newcommand{\sop}{\mathcal{S}}
\newcommand{\be}{\mathbf{e}}
\newcommand{\ww}{\mathcal{W}}
\newcommand{\subs}{\mathcal{S}ub}           %The substitutions set.
\newcommand{\vars}[1]{\mathtt{vars}(#1)}   %The set of variables.
\newcommand{\dom}[1]{\mathtt{dom}(#1)}     %The domain.
\newcommand{\ran}[1]{\mathtt{ran}(#1)}      %The range of a substitution.
\newcommand{\vran}[1]{\mathtt{v}\ran{#1}}

% Nominal Terms and Equality
\newcommand{\atomSet}{\mathbb{A}}           %The set of all atoms.
\newcommand{\var}{\mathbb{X}}               %The set of all META variables.
\newcommand{\abs}[2]{\left[ #1 \right]#2}   %Abstraction term.
\newcommand{\support}[1]{\mathtt{supp}(#1)}          %Support of an permutation.
\newcommand{\permID}{\mathtt{id}}           %Identity permutation.
\newcommand{\pAction}[2]{#1 \cdot #2}       %The action of a permutation on a term.
\newcommand{\pGroup}{\mathbb{P}}            %The group of all permutations acting on the set fo atoms.
\newcommand{\swapping}[2]{(#1 \ \ #2)}      %The Swapping permutation.
\newcommand{\terms}{T(\signature,\atomSet,\var)}    %In context of nominal this is the set of terms.

\newcommand{\fresh}{\#}
\newcommand{\tr}{\operatorname{tr}}
\newcommand{\Ric}{\mathrm{Ric}}
\newcommand{\Rm}{\mathrm{Rm}}
\newcommand{\Lip}{\mathrm{Lip}}
\newcommand{\dist}{\mathrm{dist}}
\newcommand{\Scal}{\mathrm{Scal}}

\newcommand{\secc}{\mathrm{sec}}

\newcommand{\solution}[1]{\mathcal{S}(#1)} %solution set of the problem P
\newcommand{\ueq}[2]{#1\overset{?}{=}#2} %an unification equational problem


%Notation for Equational Unification (general theory)

\newcommand{\sig}[1]{\mathcal{S}ig(#1)} %signature of a set of equations

% #1 The theory, #2 the problem set
\newcommand{\eqSolution}[2]{\mathcal{U}_{#1}(#2)}

%This defines an command with optional arguments
% The arguments are:
%m: mandatory - the theory set of equations name
%g: optional argument - CONVENTION: SEND 1 TO BE TRUE "?" means that the equation is an E-unfication problem
\DeclareDocumentCommand\eqUnif{ m g }{%
	\IfNoValueTF{#2}{=_{#1}}{\overset{?}{=}_{#1}}
}

%This defines an command with optional arguments
% The arguments are: #1 - the theory #2 - the variable set
\DeclareDocumentCommand\iqoless{ O{} O{} }{%
	\precsim_{#1}^{#2}
}
\DeclareDocumentCommand\iqogreater{ O{} O{} }{%
	\succsim_{#1}^{#2}
}
\DeclareDocumentCommand\iqoeq{ O{} O{} }{%
	\simeq_{#1}^{#2}
}

\DeclareDocumentCommand\iqoneq{ O{} O{} }{%
	\npreceq_{#1}^{#2}
}


\DeclareDocumentCommand\csu{ O{}}{%
	\mathcal{U}_{#1}
}

\newcommand{\falta}[1]{\textbf{\textcolor{red}{#1}}}
