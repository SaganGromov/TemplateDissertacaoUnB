% Configure the document math mode and theorems style.

% Comente se o trabalho for escrito em inglês
\usepackage[T1]{fontenc}
\usepackage[utf8]{inputenc}

%\usepackage[portuguese]{babel}
\usepackage[brazil]{babel}
\addto{\captionsbrazil}{%
    \renewcommand{\bibname}{Bibliografia}%
    \renewcommand{\contentsname}{Sumário}}
\listfiles
%Use this workaround the appendix error with portuguese language
\usepackage{etoolbox}
\makeatletter
\appto{\appendices}{\def\Hy@chapapp{Appendix}}
\makeatother

%Math Packages - General
\usepackage{amsmath,amsthm,amssymb,amsfonts,amscd, amsbsy,mathtools}
\usepackage{wasysym}





%Include external PDF files
\usepackage{pdfpages}


% Logic
\usepackage{bussproofs}

%\everymath{\displaystyle}
\DeclareMathAlphabet{\mathcal}{OMS}{cmsy}{m}{n}

%Pacotes de Documento
\usepackage{faktor} %Por exemplo, faktor é muito útil para escrever quocientes.
% Para mais informações veja:
% https://ctan.org/pkg/faktor?lang=en



%Ambientes Comuns - Inglês
%\theoremstyle{definition}\newtheorem{definition}{Definition}[chapter]
%\newtheorem{example}{Example}[chapter]
%\newtheorem{lemma}{Lemma}[chapter]
%\newtheorem{theorem}{Theorem}[chapter]
%\newtheorem{proposition}{Proposition}[chapter]
%\newtheorem{col}{Corollary}[chapter]
%\theoremstyle{remark}\newtheorem{remark}{Remark}[chapter]

% Ambientes Comuns - Português
%\theoremstyle{definition}\newtheorem{definicao}{Definição}[chapter]



\usepackage{amsthm}
\usepackage{pifont} 


\newcommand{\FRASE}[2]{\epigraph{\large\textit{#1}}{\textbf{#2\\}}}


\newcommand{\cmark}{\ding{51}}
\newcommand{\xmark}{\ding{55}}
\newlist{todolist}{itemize}{2}
\setlist[todolist]{label=$\square$}

\newcommand{\done}{\rlap{$\square$}{\raisebox{2pt}{\large\hspace{1pt}\cmark}}%
\hspace{-2.5pt}}
\newcommand{\wontfix}{\rlap{$\square$}{\large\hspace{1pt}\xmark}}

\definecolor{navybluegalaxy}{RGB}{0, 36, 93}
\definecolor{targ1}{RGB}{201,75,75}
\definecolor{targ2}{RGB}{60,44,44}
\definecolor{targ3}{RGB}{229,65,65}
\definecolor{targ4}{RGB}{111,68,81}
\definecolor{gold}{RGB}{255,210,0}
\definecolor{gal}{RGB}{0, 7, 111}

\newcommand{\corcaps}{gal}  
\newcommand{\COR}{navybluegalaxy}


\let\oldcitep=\citep 
\renewcommand{\citet}[1]{\textcolor{teal}{\oldcitet{#1}}}
\renewcommand{\citep}[1]{\textcolor{teal}{\oldcitep{#1}}}




\newcommand{\om}{\mathbb{M}}
\newcommand{\emvioleta}[1]{\textbf{\textcolor{violet}{#1}}}
\newcommand{\jps}[1]{\textcolor{blue}{#1}}
\newcommand{\bl}[1]{\textnormal{\textcolor{black}{#1}}}
\newcommand{\red}[1]{\textcolor{red}{#1}}
\newcommand{\white}[1]{\textcolor{white}{#1}}
\newcommand{\pur}[1]{\textcolor{purple}{#1}}
\newcommand{\maggg}[1]{\textcolor{magenta}{#1}}

\newcommand{\kozul}[4]{
\begin{aligned}
2 g(\nabla_{#1} #2, #3) &= #1(g(#2, #3)) + #2(g(#1, #3)) - #3(g(#1, #2)) \\ 
 &- g([#2, #1], #3) - g([#1, #3], #2) - g([#2, #3], #1) \\
 &#4
 \end{aligned}
}

\newcommand\cl[2]{\color{#1}{#2}}
\newcommand\bcl[2]{\color{#1}{{\fontseries{b}\selectfont #2}}}

%\usepackage{color}
%\definecolor{SAEblue}{rgb}{0, .62, .91}
%\renewcommand\theequation{\red{{\arabic{equation}}}}


\makeatletter
\let\mytagform@=\tagform@
\def\tagform@#1{\maketag@@@{\bfseries\jps{(\ignorespaces#1\unskip\@@italiccorr)}}\hspace{3mm}}
\renewcommand{\eqref}[1]{\textup{\mytagform@{\ref{#1}}}}
\makeatother

\renewcommand{\div}{\divv}
\usepackage[LGR,T1]{fontenc}
\newcommand{\textgreek}[1]{\begingroup\fontencoding{LGR}\selectfont#1\endgroup}


\newcommand{\KN}{\mathbin{\bigcirc\mspace{-15mu}\wedge\mspace{4mu}}}
\theoremstyle{theorem}\newtheorem{obs}{Observação}[chapter]
\renewcommand{\qedsymbol}{\rule{0.7em}{0.7em}}
\newenvironment{demm}{\smallskip \noindent{\bf \underline{Demonstração:}}}
{\begin{flushright} $\qedsymbol$\end{flushright}\smallskip}

\newenvironment{demm41}{\smallskip \noindent{\bf \underline{Demonstração do Teorema \cref{maxxnumeig}:}}}
{\begin{flushright} $\qedsymbol$\end{flushright}\smallskip}


\newenvironment{demm42}{\smallskip \noindent{\bf \underline{Demonstração do Teorema \cref{localstruc}:}}}
{\begin{flushright} $\qedsymbol$\end{flushright}\smallskip}


\newtheoremstyle{theoremdd}% name of the style to be used
  {\topsep}% measure of space to leave above the theorem. E.g.: 3pt
  {\topsep}% measure of space to leave below the theorem. E.g.: 3pt
  {}% name of font to use in the body of the theorem
  {1pt}% measure of space to indent
  {\bfseries\color{violet}}% name of head font
  {}% punctuation between head and body
  { }% space after theorem head; " " = normal interword space
  {\underline{\thmname{#1} (\thmnumber{O.#2})\textbf{\thmnote{ (#3)}.}}}


  
\theoremstyle{theoremdd}
\newtheorem{oobs}{Observação}

\newtheoremstyle{theorempergunta}% name of the style to be used
  {\topsep}% measure of space to leave above the theorem. E.g.: 3pt
  {\topsep}% measure of space to leave below the theorem. E.g.: 3pt
  {}% name of font to use in the body of the theorem
  {1pt}% measure of space to indent
  {\bfseries\color{gal}}% name of head font
  {}% punctuation between head and body
  { }% space after theorem head; " " = normal interword space
  {\underline{\thmname{#1} (\thmnumber{P.#2})\textbf{\thmnote{ (#3)}.}}}


\theoremstyle{theorempergunta}
\newtheorem{pergunta}{Pergunta}

\newtheoremstyle{theoremconjec}% name of the style to be used
  {\topsep}% measure of space to leave above the theorem. E.g.: 3pt
  {\topsep}% measure of space to leave below the theorem. E.g.: 3pt
  {}% name of font to use in the body of the theorem
  {1pt}% measure of space to indent
  {\bfseries\color{gal}}% name of head font
  {}% punctuation between head and body
  { }% space after theorem head; " " = normal interword space
  {\underline{\thmname{#1.} \thmnumber{}\textbf{\thmnote{}}}}

\theoremstyle{theoremconjec}
\newtheorem{conjec}{A conjectura da geometrização de Thurston}

\newtheoremstyle{theoremEX}% name of the style to be used
  {\topsep}% measure of space to leave above the theorem. E.g.: 3pt
  {\topsep}% measure of space to leave below the theorem. E.g.: 3pt
  {}% name of font to use in the body of the theorem
  {1pt}% measure of space to indent
  {\bfseries\color{BlueViolet}}% name of head font
  {}% punctuation between head and body
  { }% space after theorem head; " " = normal interword space
  {\underline{\thmname{#1} (\thmnumber{E.#2})\textbf{\thmnote{ (#3)}.}}}

\theoremstyle{theoremEX}
\newtheorem{exem}{Exemplo}[chapter]

\theoremstyle{theoremNormal}





\newtheoremstyle{theoremDEF}% name of the style to be used
  {\topsep}% measure of space to leave above the theorem. E.g.: 3pt
  {\topsep}% measure of space to leave below the theorem. E.g.: 3pt
  {}% name of font to use in the body of the theorem
  {1pt}% measure of space to indent
  {\bfseries\color{cyan}}% name of head font
  {}% punctuation between head and body
  { }% space after theorem head; " " = normal interword space
  {\underline{\thmname{#1} (\thmnumber{D.#2})\textbf{\thmnote{ (#3)}.}}}

\theoremstyle{theoremDEF}
\newtheorem{deff}{Definição}[chapter]

\newtheoremstyle{theoremPROP}% name of the style to be used
  {\topsep}% measure of space to leave above the theorem. E.g.: 3pt
  {\topsep}% measure of space to leave below the theorem. E.g.: 3pt
  {}% name of font to use in the body of the theorem
  {1pt}% measure of space to indent
  {\bfseries\color{Sepia}}% name of head font
  {}% punctuation between head and body
  { }% space after theorem head; " " = normal interword space
  {\underline{\thmname{#1} (\thmnumber{P.#2})\textbf{\thmnote{ (#3)}.}}}

\theoremstyle{theoremPROP}
\newtheorem{proposicao}{Proposição}[chapter]

\newtheoremstyle{theoremLEM}% name of the style to be used
  {\topsep}% measure of space to leave above the theorem. E.g.: 3pt
  {\topsep}% measure of space to leave below the theorem. E.g.: 3pt
  {}% name of font to use in the body of the theorem
  {1pt}% measure of space to indent
  {\bfseries\color{navybluegalaxy}}% name of head font
  {}% punctuation between head and body
  { }% space after theorem head; " " = normal interword space
  {\underline{\thmname{#1} (\thmnumber{L.#2})\textbf{\thmnote{ (#3)}.}}}

\theoremstyle{theoremLEM}
\newtheorem{lema}{Lema}[chapter]


\newtheoremstyle{theoremTEO}% name of the style to be used
  {\topsep}% measure of space to leave above the theorem. E.g.: 3pt
  {\topsep}% measure of space to leave below the theorem. E.g.: 3pt
  {\itshape}% name of font to use in the body of the theorem
  {5pt}% measure of space to indent
  {\bfseries\color{gal}}% name of head font
  {}% punctuation between head and body
  { }% space after theorem head; " " = normal interword space
  {\underline{\thmname{#1} (\thmnumber{T.#2})\textbf{\thmnote{ (#3)}.}}}
  


\theoremstyle{theoremTEO}
\newtheorem{teorema}{Teorema}[chapter]

\newtheoremstyle{theoremCOL}% name of the style to be used
  {\topsep}% measure of space to leave above the theorem. E.g.: 3pt
  {\topsep}% measure of space to leave below the theorem. E.g.: 3pt
  {\itshape}% name of font to use in the body of the theorem
  {5pt}% measure of space to indent
  {\bfseries\color{Aquamarine}}% name of head font
  {}% punctuation between head and body
  { }% space after theorem head; " " = normal interword space
  {\underline{\thmname{#1} (\thmnumber{C.#2})\textbf{\thmnote{ (#3)}.}}}
  
\theoremstyle{theoremCOL} 
\newtheorem{col}{Corolário}[chapter]
\newcommand{\mm}{\mathcal{M}}
\newcommand{\nn}{\mathcal{N}}
\renewcommand{\epsilon}{\varepsilon}
