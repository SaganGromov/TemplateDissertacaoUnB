% ******************************* Thesis Dedidcation ********************************
\newgeometry{bottom=0.1in,top=0in,left=0.3in,right=0.3in}
\begin{dedication}

\vspace{7.1cm}
%\PlaceText{69mm}{11mm}{ \color{gal}\noindent\makebox[\linewidth]{\rule{2\paperwidth}{1pt}}}
\PlaceText{69mm}{12.5mm}{ \color{\COR}\noindent\makebox[\linewidth]{\rule{2\paperwidth}{1pt}}}
\PlaceText{13mm}{79}{\huge{\textbf{\textcolor{gal}{Dedico esse trabalho a Carl Edward Sagan.}}}}
\PlaceText{69mm}{33mm}{ \color{\COR}\noindent\makebox[\linewidth]{\rule{2\paperwidth}{1pt}}}


\begin{textblock*}{5cm}(0.5cm,4cm)

\epigraph{\large\textit{What an astonishing thing a book is. It's a flat object made from a tree with flexible parts on which are imprinted lots of funny dark squiggles. But one glance at it and you're inside the mind of another person, maybe somebody dead for thousands of years. Across the millennia, an author is speaking clearly and silently inside your head, directly to you. Writing is perhaps the greatest of human inventions, binding together people who never knew each other, citizens of distant epochs. Books break the shackles of time. A book is proof that humans are capable of working magic.}}{\textbf{Carl Sagan \\}}


\epigraph{\large\textit{We are a way for the Cosmos to know itself.}}{\textbf{Carl Sagan \\}}


\epigraph{\large\textit{If you wish to make an apple pie from scratch, you must first invent the universe.}}{\textbf{Carl Sagan \\}}





\end{textblock*}

\begin{textblock*}{5cm}(12.5cm,4cm)

\epigraph{\large\textit{Extraordinary claims require extraordinary evidence.}}{\textbf{Carl Sagan \\}}


\epigraph{\large\textit{Who is more humble? The scientist who looks at the universe with an open mind and accepts whatever the universe has to teach us, or somebody who says everything in this book must be considered the literal truth and never mind the fallibility of all the human beings involved?}}{\textbf{Carl Sagan \\}}

\epigraph{\large\textit{Imagination will often carry us to worlds that never were. But without it we go nowhere.}}{\textbf{Carl Sagan \\}}

\epigraph{\large\textit{It pays to keep an open mind, but not so open your brains fall out.}}{\textbf{Carl Sagan \\}}

\epigraph{\large\textit{For me, it is far better to grasp the Universe as it really is than to persist in delusion, however satisfying and reassuring.}}{\textbf{Carl Sagan \\}}


\iffalse

\epigraph{\large\textit{One of the saddest lessons of history is this: If we’ve been bamboozled long enough, we tend to reject any evidence of the bamboozle. We’re no longer interested in finding out the truth. The bamboozle has captured us. It’s simply too painful to acknowledge, even to ourselves, that we’ve been taken. Once you give a charlatan power over you, you almost never get it back.}}{\textbf{Carl Sagan \\}}

\fi


\end{textblock*}





\newpage

\newgeometry{bottom=0.1in,top=0in,left=0.5in,right=0.5in}
\vbox{
\begin{textblock*}{5cm}(12.5cm,0cm) % {block width} (coords) 
%
\epigraph{\large\textit{Aut viam inveniam aut faciam.}}{\textbf{Hannibal Barca \\}}

\epigraph{\large\textit{Only those who will risk going too far can possibly find out how far one can go.}}{\textbf{T. S. Elliot \\}}


\epigraph{\large\textit{There is no royal road to geometry.}}{\textbf{Euclid \\}}

\epigraph{\large\textit{We are absurdly accustomed to the miracle of a few written signs being able to contain immortal imagery, involutions of thought, new worlds with live people, speaking, weeping, laughing. We take it for granted so simply that in a sense, by the very act of brutish routine acceptance, we undo the work of the ages, the history of the gradual elaboration of poetical description and construction, from the treeman to Browning, from the caveman to Keats. What if we awake one day, all of us, and find ourselves utterly unable to read? I wish you to gasp not only at what you read but at the miracle of its being readable.
}}{\textbf{Vladimir Nabokov \\}}



\end{textblock*}

\begin{textblock*}{5cm}(0.5cm,0cm) % {block width} (coords) 

\epigraph{\large\textit{Mathematics is not about numbers, equations, computations, or algorithms: it is about understanding.}}{\textbf{William Thurston \\}}

%\epigraph{\large\textit{Aut viam inveniam aut faciam.}}{\textbf{Hannibal Barca \\}}
\iffalse
\epigraph{\textit{Imagination will often carry us to worlds that never were. But without it we go nowhere.}}{\textbf{Carl Sagan \\}}
\fi

\iffalse
\epigraph{\large\textit{If people do not realize that mathematics is simple, it is only because they do not realize how complicated life is.}}{\textbf{John von Neumann\\}}
\fi

\epigraph{\large\textit{Wir müssen wissen. \newline Wir werden wissen. }}{\textbf{David Hilbert\\}}

\epigraph{\large\textit{If I have seen further it is only by standing on the shoulders of giants.}}{\textbf{Isaac Newton\\}}


\FRASE{A human being is part of a whole, called by us the \quotes{Universe}, a part limited in time and space. He experiences himself, his thoughts and feelings, as something separated from the rest - a kind of optical delusion of his consciousness. This delusion is a kind of prison for us, restricting us to our personal desires and to affection for a few persons nearest us. Our task must be to free ourselves from this prison by widening our circles of compassion to embrace all living creatures and the whole of nature in its beauty.}{Albert Einstein}

%\epigraph{\textit{Great men are forged in fire. It is the privillege of lesser men to light the flame.}}{\textbf{Isaac Newton\\}}
\PlaceText{3mm}{799}{\textcolor{gal}{\large\textbf{\textit{Great men are forged in fire. It is the privilege of lesser men to light the flame.}}}}

\PlaceText{85mm}{821}{\textcolor{gal}{\small\textbf{\textit{Steven Moffat}}}}







\iffalse
\epigraph{\textit{The Communists disdain to conceal their views and aims. They openly declare that their ends can be attained only by the forcible overthrow of all existing social conditions. Let the ruling classes tremble at a Communistic revolution. The proletarians have nothing to lose but their chains. They have a world to win. Working Men of All Countries, Unite!}}{\textbf{Karl Marx\\}}
\fi







\end{textblock*}
\vspace{-6cm}



\iffalse
\epigraph{\large\textit{Only those who will risk going too far can possibly find out how far one can go.}}{\textbf{T. S. Elliot \\}}

\epigraph{\large\textit{What an astonishing thing a book is. It's a flat object made from a tree with flexible parts on which are imprinted lots of funny dark squiggles. But one glance at it and you're inside the mind of another person, maybe somebody dead for thousands of years. Across the millennia, an author is speaking clearly and silently inside your head, directly to you. Writing is perhaps the greatest of human inventions, binding together people who never knew each other, citizens of distant epochs. Books break the shackles of time. A book is proof that humans are capable of working magic.}}{\textbf{Carl Sagan \\}}

\epigraph{\large\textit{Extraordinary claims require extraordinary evidence.}}{\textbf{Carl Sagan \\}}
\epigraph{\large\textit{Mathematics is not about numbers, equations, computations, or algorithms: it is about understanding.}}{\textbf{William Thurston \\}}
\fi
}
\end{dedication}
\restoregeometry
